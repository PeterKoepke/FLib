\newtheorem{signature}{Signature}
\newtheorem{axiom}{Axiom}
\newtheorem{signaturep}{Signature}
\newtheorem{axiomp}{Axiom}
\newtheorem{definitionp}{Definition}
\newtheorem{theoremp}{Theorem}
\newtheorem{lemmap}{Lemma}
 
\newcommand{\power}{{\cal P}} 
\newcommand{\preimg}[2]{{#1}^{-1}[#2]} 
\newcommand{\Seq}[2]{\{#1,\dots,#2\}}
\newcommand{\Set}[3]{\{#1_{#2},\dots,#1_{#3}\}}
\newcommand{\Product}[3]{\prod_{i=#2}^{#3}{#1}_i}
\newcommand{\subfunc}[2]{{#1}_{#2}}
\newcommand{\CC}{{\Bbb C}}
\newcommand{\RR}{{\Bbb R}}
\newcommand{\QQ}{{\Bbb Q}}
\newcommand{\ZZ}{{\Bbb Z}} 
\newcommand{\NN}{{\Bbb N}}
\newcommand{\NNplus}{{\Bbb N}^+}

[prove off]

\title{There are infinitely many primes}
\maketitle
%#
%# Version 18 March 2012
%#
\section{I. Foundations}

\subsection{1. Sets}

%%[set/-s] [element/-s] [belong/-s] [subset/-s]

\begin{signature} SetSort.  A {\bf set} is a notion.
Let $A,X,Y,Z$ denote sets.
\end{signature}

\begin{signature} ElmSort.  
An {\bf element} is a notion.
Let $x,y,z$ denote elements.
\end{signature}

\begin{signature} EOfElem.  
An {\bf element of} $X$ is an element.
Let $x$ belongs to $X$ stand for $x$ is an element of $X$.
Let $x$ is in $X$ stand for $x$ belongs to $X$.
Let $x \in X$ stand for $x$ is in $X$.
\end{signature}

\begin{definition} DefEmp.
$\emptyset$ is a set that has no elements.
Let $X$ is {\bf empty} stand for $X = \emptyset$.
Let $X$ is {\bf nonempty} stand for $X \neq \emptyset$.
\end{definition}

\begin{definition} DefSub.  
A {\bf subset} of $Y$ is a set $X$ such that
every element of $X$ belongs to $Y$.
Let $X \subseteq Y$ stand for $X$ is a subset of $Y$.
\end{definition}

\begin{lemma} SubRefl.
$X \subseteq X$.
\end{lemma}

\begin{lemma} SubTrans.
$X \subseteq Y \subseteq Z  \rightarrow  X \subseteq Z$.
\end{lemma}

\begin{axiom} SubASymm.
$X \subseteq Y \subseteq X  \rightarrow  X = Y$.
\end{axiom}


\subsection{2. Functions}

%%[function/-s]

\begin{signature} FunSort.
A {\bf function} is a notion.
Let $f,g,p$ denote functions.
\end{signature}

\begin{signature} DomSet.
$\text{dom} f$ is a set.
Let the {\bf domain} of $f$ stand for $\text{dom} f$.
\end{signature}

\begin{signature} ImgElm.
Let $x \in \text{dom} f$. $f(x)$ is an element.
Let $\subfunc{f}{x}$ stand for $f(x)$.
\end{signature}

\begin{definition} DefSImg. 
Let $X \subseteq \text{dom} f$. $f[X] = \{ f(x) | x \in X \}$.
Let $\text{ran} f$ stand for $f[\text{dom} f]$.
Let the {\bf range} of $f$ stand for $\text{ran} f$.

Let a function {\bf from} $X$ stand for a function $f$
such that $\text{dom} f = X$.
Let a function {\bf from} $X$ {\bf to} $Y$ stand 
for a function $f$ such that $\text{dom} f = X$
and $\text{ran} f \subseteq Y$.

Let $f : X$ stand for $f$ is a function from $X$.
Let $f : X -> Y$ stand for $f$ is a function from $X$ to $Y$.
\end{definition}

\begin{lemma} ImgRng.
Let $x \in \text{dom} f$. $f(x)$ belongs to $\text{ran} f$.
\end{lemma}

\begin{definition} DefRst.
Let $X \subseteq \text{dom} f$. 
$f \upharpoonright X$ is a function $g$ from $X$
such that for every $x \in X$ $g(x) = f(x)$.
\end{definition}

\subsection{3. Numbers}

%%[number/-s]

\begin{signature} NatSort.
A {\bf number} is a notion.
Let i,j,k,l,m,n,q,r denote numbers.
\end{signature}

\begin{definition} Nat.
$\NN$ is the set of numbers.
\end{definition}

\begin{signature} SortsC.  
$0$ is a number.
Let $x$ is {\bf nonzero} stand for $x \neq 0$.
\end{signature}

\begin{signature} SortsC.
$1$ is a nonzero number.
\end{signature}

\begin{signature} SortsB.
$m + n$ is a number.
\end{signature}

\begin{signature} SortsB.
$m \cdot n$ is a number.
\end{signature}

\begin{axiom} AddAsso. $(m + n) + l = m + (n + l)$.\end{axiom}
\begin{axiom} AddZero.  $m + 0 = m = 0 + m$. \end{axiom}
\begin{axiom} AddComm.   $m + n = n + m$. \end{axiom}

\begin{axiom} MulAsso.  
$(m \cdot n) \cdot l = m \cdot (n \cdot l)$.
\end{axiom}
\begin{axiom} MulUnit.  $m \cdot 1 = m = 1 \cdot m$.\end{axiom}
\begin{axiom} MulZero.  $m \cdot 0 = 0 = 0 \cdot m$.\end{axiom}
\begin{axiom} MulComm.  $m \cdot n = n \cdot m$.\end{axiom}

\begin{axiom} AMDistr.  
$m \cdot (n + l) = (m \cdot n) + (m \cdot l)$ and
$(n + l) \cdot m = (n \cdot m) + (l \cdot m)$.\end{axiom}

\begin{axiom} AddCanc.  
If $l + m = l + n$ or $m + l = n + l$ then $m = n$.\end{axiom}

\begin{axiom} MulCanc.
Assume that $l$ is nonzero. If 
$l \cdot m = l \cdot n$ or $m \cdot l = n \cdot l$ 
then $m = n$.\end{axiom}

\begin{axiom} ZeroAdd.
If $m + n = 0$ then $m = 0$ and $n = 0$.\end{axiom}

\begin{lemma} ZeroMul.
If $m \cdot n = 0$ then $m = 0$ or $n = 0$.
\end{lemma}

\begin{definition} DefLE.
$m \leq n$ iff there exists $l$ such that $m + l = n$.
\end{definition}

\begin{definition} DefDiff.  Assume that $m \leq n$.
$n - m$ is a number $l$ such that $m + l = n$.
\end{definition}

\begin{lemma} LERefl. $m \leq m$. \end{lemma}
\begin{lemma} LETran. $m \leq n \leq l  \rightarrow  m \leq l$.
\end{lemma}
\begin{lemma} LEAsym. $m \leq n \leq m  \rightarrow  m = n$. 
\end{lemma}

Let $m < n$ stand for $m \neq n$ and $m \leq n$.

\begin{axiom} LETotal. $m \leq n$ or $n < m$. \end{axiom}

\begin{lemma} MonAdd. Assume that $l < n$.
Then $m + l < m + n$ and $l + m < n + m$.
\end{lemma}

\begin{lemma} MonMul. Assume that $m$ is nonzero and $l < n$.
Then $m \cdot l < m \cdot n$ and $l \cdot m < n \cdot m$.
\end{lemma}

\begin{axiom} LENTr. 
$n = 0$ or $n = 1$ or $1 < n$.\end{axiom}


\begin{lemma} MonMul2. $m \neq 0 \rightarrow n \leq n \cdot m$.
\end{lemma}
\begin{proof}
Let $m \neq 0$. Then $1 \leq m$.
\end{proof}

\begin{signature} InbuiltForthelInductionRelation. $m \precprec n$ is an atom.
\end{signature}

\begin{axiom} EmbeddingLessIntoInductionRelation. $m < n \rightarrow m \precprec n$. \end{axiom}

\subsection{4. Finite Sets and Sequences}

\begin{definitionp} 
$\Seq{m}{n} = \{ i \in \NN | m \leq i \leq n \}$.
\end{definitionp}

\begin{definitionp} 
Let $f$ be a function such that 
$\Seq{m}{n} \subseteq \text{dom} f$. 
$\Set{f}{m}{n} = \{ f(i) | i \in \NN \wedge m \leq i \leq n \}$.
\end{definitionp}

\begin{definitionp} $f$ {\bf lists} $X$ in $n$ steps iff 
$\text{dom} f = \Seq{1}{n}$ and $X = \Set{f}{1}{n}$.
\end{definitionp}

\begin{definitionp} $X$ is {\bf finite} iff there is a function
$f$ and a number $n$ such that $f$ lists $X$ in $n$ steps.
\end{definitionp}

\begin{definitionp} $X$ is {\bf infinite} iff $X$ is not finite.
\end{definitionp}

[/prove]

[sequence/-s]

Signature. A finite sequence is a notion.
Let $u,v,w,x,y,z$ stand for finite sequences.

Signature. The length of u is a number.
Let $|u|$ stand for the length of $u$.

Signature. Let $1 \leq i \leq |u|$. $u(i)$ is an element.

Let $\Set{u}{1}{n}$ stand for $u$ is a finite sequence and $|u| = n$.

\begin{signaturep}
$u ++ v$ is a finite sequence $w$ such that
$|w| = |u| + |v|$
and for every number $i$ such that $1 \leq i \leq |u|$ we have 
$w(i) = u(i)$
and for every number $j$ such that 
$|u|+1 \leq j \leq |u|+|v|$ we have 
$w(j) = v(j-|u|)$.
\end{signaturep}

\begin{lemma} Length1.
$|(u++v)++w| = (|u|+|v|)+|w|$.
\end{lemma}

\begin{lemma} Length2.
$|u++(v++w)| = |u|+(|v|+|w|)$.
\end{lemma}

\begin{lemmap}
$|(u++v)++w| = |u++(v++w)|$ (by Length1, Length2, AddAsso).
\end{lemmap}

\begin{lemma}.
Let $x = (u++v)++w$. Let $y = u++(v++w)$. 
Let $m = |x|$.
For every number $i$ such that $1 \leq i \leq m$ we have
$x(i)=y(i)$.
\end{lemma}
\begin{proof}
Let $j = |u|$. Let $k = |v|$. Let $l = |w|$.
Let $1 \leq i \leq m$.
Case $1 \leq i \leq j$.
Then
$x(i) = (u++v)(i) = u(i) = y(i)$. end.
\end{proof}


