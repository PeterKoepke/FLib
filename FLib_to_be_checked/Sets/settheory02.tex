\documentclass{article}
\usepackage[english]{babel}
\usepackage{enumerate, latexsym, amssymb, amsmath}
\usepackage{framed, multicol}
\newenvironment{forthel}{\begin{leftbar}}{\end{leftbar}}

%%%%%%%%%% Start TeXmacs macros
\newcommand{\tmaffiliation}[1]{\\ #1}
\newcommand{\tmem}[1]{{\em #1\/}}
\newenvironment{enumeratenumeric}{\begin{enumerate}[1.] }{\end{enumerate}}
\newenvironment{proof}{\noindent\textbf{Proof\ }}{\hspace*{\fill}$\Box$\medskip}
\newenvironment{quoteenv}{\begin{quote} }{\end{quote}}
\newtheorem{axiom}{Axiom}
\newtheorem{lemma}{Lemma}
\newtheorem{theorem}{Theorem}
\newtheorem{definition}{Definition}
\newtheorem{signature}{Signature}
\newtheorem{proposition}{Proposition}
%%%%%%%%%% End TeXmacs macros

\newcommand{\event}{UITP 2018}
\newcommand{\dom}{Dom}
\newcommand{\fun}{aFunction}
\newcommand{\sym}{sym}
\newcommand{\halfline}{{\vspace{3pt}}}
\newcommand{\tab}{{\hspace{1cm}}}
\newcommand{\ball}[2]{B_{#1}(#2)}
\newcommand{\llbracket}{[}
\newcommand{\rrbracket}{]}
\newcommand{\less}[1]{<_{#1}}
\newcommand{\greater}[1]{>_{#1}}
\newcommand{\leeq}[1]{{\leq}_{#1}}
\newcommand{\supr}[1]{\mathrm{sup}_{#1}}
\newcommand{\RR}{\mathbb{R}}
\newcommand{\QQ}{\mathbb{Q}}
\newcommand{\ZZ}{\mathbb{Z}}
\newcommand{\NN}{\mathbb{N}}

\begin{document}

\title{Set Theory}

\author{Peter Koepke}

\date{August 25, 2018}

\maketitle


\section{Preliminaries}

\begin{forthel}
Let $x,y,z$ stand for sets.
Let $x \in y$ denote $x$ is an element of $y$.
%# Let $x$ is \emph{in} $y$ denote $x$ is an element of $y$.
Let $x \notin y$ denote $x$ is not an element of $y$.

\begin{axiom} $(x \in y) \Rightarrow x -<- y$.
\end{axiom} 

\begin{theorem} $x \notin x$.
\end{theorem}
{\bf Proof} by induction on $x$. {\bf qed.}

\begin{signature} The \emph{empty set} is the set that has no elements.
Let $\emptyset$ denote the empty set.
\end{signature}

\begin{definition} $x$ is \emph{nonempty} iff $x$ has an element.
\end{definition}

\begin{definition} A subset of $y$ is a set $x$ such that every element 
of $x$ is an element of $y$. 
Let $x \subseteq y$ stand for $x$ is a subset of $y$.
Let $y \supseteq x$ stand for $x$ is a subset of $y$.
\end{definition}

\begin{definition} A \emph{proper subset} of $y$ is a subset $x$ of $y$ such that there is an element of $y$ that is not an element of $x$.
\end{definition}

\begin{proposition} $x \subseteq x$. \end{proposition}

\begin{proposition} If $x \subseteq y \subseteq x$ then $x = y$. \end{proposition}

\begin{definition} $x \cup y = \{u \mid u \in x \text{ or } u \in y\}$. \end{definition}

\begin{definition} $\{x\} = \{\text{sets }y \mid y = x\}$.\end{definition}
\end{forthel}

\section{Ordinal Numbers}

\begin{forthel}
[ordinal/-s]

\begin{definition} $x$ is \emph{transitive} iff
every element of $x$ is a subset of $x$.
Let $\rm{Trans}(x)$ stand for $x$ is transitive.
\end{definition}

\begin{definition} 
An \emph{ordinal} is a set $x$ such that $x$ is transitive and every element of $x$ is transitive.
Let $\rm{Ord}(x)$ stand for $x$ is an ordinal.
Let $\alpha, \beta, \gamma$ stand for ordinals.
\end{definition}

\begin{theorem} $\emptyset$ is an ordinal.
\end{theorem}

\begin{definition} $x + 1 = x \cup \{x\}$.
\end{definition}

\begin{theorem}
$\alpha + 1$ is an ordinal.
\end{theorem}

\begin{theorem} If $\rm{Ord}(\beta)$ and $\alpha \in \beta$ then
$\rm{Ord}(\alpha)$.
\end{theorem}

\begin{theorem} $(\alpha \in \beta) \wedge (\beta \in \gamma) 
\Rightarrow (\alpha \in \gamma)$.
\end{theorem}

\end{forthel}

\end{document}