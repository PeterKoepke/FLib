\documentclass{scrartcl}
\usepackage[english]{babel}
\usepackage{enumerate, latexsym, amssymb, amsmath, amsthm}
\usepackage{framed, multicol}
\newenvironment{forthel}{\begin{leftbar}}{\end{leftbar}}

%%%%%%%%%% Start TeXmacs macros
\newcommand{\tmaffiliation}[1]{\\ #1}
\newcommand{\tmem}[1]{{\em #1\/}}
\newenvironment{enumeratenumeric}{\begin{enumerate}[1.] }{\end{enumerate}}
%\newenvironment{proof}{\noindent\textbf{Proof\ }}{\hspace*{\fill}$\Box$\medskip}
\newenvironment{quoteenv}{\begin{quote} }{\end{quote}}
\newtheorem*{axiom}{Axiom}
\newtheorem*{lemma}{Lemma}
\newtheorem*{theorem}{Theorem}
\newtheorem*{definition}{Definition}
\newtheorem*{signature}{Signature}
\newtheorem*{proposition}{Proposition}
%%%%%%%%%% End TeXmacs macros

\newcommand{\event}{UITP 2018}
\newcommand{\dom}{Dom}
\newcommand{\fun}{aFunction}
\newcommand{\sym}{sym}
\newcommand{\halfline}{{\vspace{3pt}}}
\newcommand{\tab}{{\hspace{1cm}}}
\newcommand{\ball}[2]{B_{#1}(#2)}
\newcommand{\llbracket}{[}
\newcommand{\rrbracket}{]}
\newcommand{\sing}[1]{\{{#1}\}}
\newcommand{\pair}[2]{\{{#1},{#2}\}}
\newcommand{\op}[2]{({#1},{#2})}
\newcommand{\less}[1]{<_{#1}}
\newcommand{\greater}[1]{>_{#1}}
\newcommand{\leeq}[1]{{\leq}_{#1}}
\newcommand{\supr}[1]{\mathrm{sup}_{#1}}
\newcommand{\RR}{\mathbb{R}}
\newcommand{\QQ}{\mathbb{Q}}
\newcommand{\ZZ}{\mathbb{Z}}
\newcommand{\NN}{\mathbb{N}}

\begin{document}

\title{ELEMENTARY SET THEORY}

\subtitle{An SAD3 Formalisation of the Appendix of \\"General Topology" 
by John L. Kelley \\
Functions and Preliminaries}

\date{November 3, 2018}

\maketitle


\subsection{Preliminaries}

\begin{forthel}

[prove off]

Let $x,y,z$ stand for \emph{classes}.


[object/-s]
\begin{signature}[Ontology] An object is a notion.
Let $a,b,c,d,e,u,v$ stand for objects.
\end{signature}

Let $a \in x$ stand for $a$ is an \emph{element} of $x$.

\begin{axiom} Every element of $x$ is an object. \end{axiom}

\begin{axiom}[I]
For each $x$ for each $y$ $x = y$ iff for each $z$ 
$z \in x$ iff $z \in y$.
\end{axiom}

[set/-s]
\begin{definition}[1] A \emph{set} is a class that is an object.
\end{definition}


\begin{definition}[2] $x \cup y = \{\text{object } u \mid u \in x \text{ or } u \in y \}$.
\end{definition}

\begin{definition}[23] $\bigcup x = 
\{object u \mid \text{ for some } y (y \in x \text{ and } u \in y)\}$.
Let the \emph{union} of $x$ stand for $\bigcup x$.
\end{definition}

\begin{definition}[25] 
A \emph{subclass} of $y$ is a class $x$ such that each element of $x$ is an
element of $y$. Let $x \subset y$ stand for $x$ is a subclass of $y$.
Let $x$ is \emph{contained} in $y$ stand for $x \subset y$.
\end{definition}

\begin{theorem}[27] $x = y$ iff $x \subset y$ and $y \subset x$.
\end{theorem}

\begin{theorem}[28] If $x \subset y$ and $y \subset z$ then $x \subset z$.
\end{theorem}

\begin{axiom}[III] If $x$ is a set then there is a set $y$ such that for each
$z$ if $z \subset x$ then $z \in y$.
\end{axiom}

\begin{theorem}[33] If $x$ is a set and $z \subset x$ then $z$ is a set.
\end{theorem}

\begin{definition}[36] $2^{x} = \{\text{set } y \mid y \subset x\}$.
\end{definition}

\begin{theorem}[38a] If $x$ is a set then $2^{x}$ is a set.
\end{theorem}
\begin{proof} Let $x$ be a set.
Take a set $y$ such that for each $z$ 
if $z \subset x$ then $z \in y$ (by III).
Then $2^{x} \subset y$.
\end{proof}

\begin{definition}[40] $\sing{a} = \{a\}$.
\end{definition}

\begin{signature}[48] $\op{a}{b}$ is an object.
\end{signature}

\begin{definition}[48a]
An \emph{ordered pair} is an object $c$ such that there exist
objects $a$ and $b$ such that $c = \op{a}{b}$.
\end{definition}

\begin{axiom}[55] If $\op{a}{b}=\op{c}{d}$ then
$a = c$ and $b = d$.
\end{axiom}


[relation/-s]
\begin{definition}[56] A \emph{relation} is a class $r$ such that
every element of $r$ is an ordered pair.
\end{definition}

Let $r,s,t$ stand for relations.

\begin{definition}[57] 
$r \circ s =
\{\op{x}{z} \mid x,z \text{ are objects and there exists } b \text{ such that }
\op{x}{b} \in s \text{ and } \op{b}{z} \in r\}$.
\end{definition}


\end{forthel}

\subsection{Functions (Maps)}

Since "function" is predefined in SAD3, we use the word "map" instead.

\begin{forthel}
[/prove]
[map/-s]
\begin{definition}[63] A \emph{map} is a relation $f$ such that for each
$a,b,c$ if $\op{a}{b} \in f$ and $\op{a}{c} \in f$ then $b = c$.
Let $f,g$ stand for maps.
\end{definition}

\begin{theorem}[64] If $f, g$ are maps then $f \circ g$ is a map.
\end{theorem}

\begin{definition}[65] ${\rm domain} f = \{\text{object } u \mid 
\text{ there exists an object } v \text{ such that } \op{u}{v} \in f\}$.
\end{definition}

\begin{definition}[66] ${\rm range} f =  \{\text{object } v \mid 
\text{ there exists an object } u \text{ such that } \op{u}{v} \in f\}$.
\end{definition}

\begin{signature}[68] Let $u \in {\rm domain} f$.
The value of $f$ at $u$ is an object $v$ such that $\op{u}{v} \in f$.
Let $f(u)$ stand for the value of $f$ at $u$.
\end{signature}

\begin{theorem}[70] Let $f$ be a map. Then 
$f = \{\op{u}{f(u)} \mid u \in {\rm domain} f\}$.
\end{theorem}

\begin{theorem}[71] 
Assume ${\rm domain} f = {\rm domain} g$ and
for every element $u$ of ${\rm domain} f$  $f(u) = g(u)$.
Then $f = g$.
\end{theorem}

\begin{axiom}[V] If $f$ is a map and ${\rm domain} f$ is a set 
then ${\rm range} f$ is a set.
\end{axiom}

\begin{axiom}[VI] If $x$ is a set then $\bigcup x$ is a set.
\end{axiom}

\begin{definition}[72] $x \times y = 
\{\op{u}{v} \mid u \in x \text{ and } v \in y\}$.
\end{definition}

\begin{theorem}[73] Let $u$ be an object. Let $y$ be a set.
Then $\sing{u} \times y$ is a set.
\end{theorem}
\begin{proof} Define
$f = \{\op{w}{v} \mid w \in y \text{ and } v = \op{u}{w}\}$.
$f$ is a map. 
${\rm domain} f = y$.
${\rm range} f = \sing{u} \times y$.
\end{proof}

\begin{theorem}[74] Let $x,y$ be sets. Then $x \times y$ is a set.
\end{theorem}
\begin{proof} 
Define $f = \{\op{u}{w} \mid u \in x \text{ and } w = \sing{u} \times y\}$.
$f$ is a map.
${\rm domain} f = x$.
${\rm range} f$ is a set.
${\rm range} f = \{\text{set } z \mid 
\text{there exists } u \in x \text{ such that } z = \sing{u} \times y\}$.
$\bigcup ({\rm range} f)$ is a set.
$\bigcup ({\rm range} f) \subset x \times y$.
Let us show that $x \times y \subset \bigcup ({\rm range} f)$.
Let $w \in x \times y$. Take 
an $u \in x$ and $v \in y$ such that $w = \op{u}{v}$.
$w \in \sing{u} \times y \in {\rm range} f$.
$w \in \bigcup {\rm range} f$.
end.
\end{proof}

\begin{theorem}[75] Let $f$ be a map. Let ${\rm domain} f$ be a set.
Then $f$ is a set.
\end{theorem}
\begin{proof}
$f \subset {\rm domain} f \times {\rm range} f$.
\end{proof}

\begin{definition}[76]
$y^x = \{\text{map } f \mid 
{\rm domain} f = x \text{ and } {\rm range} f \subset y\}$.
\end{definition}

\begin{theorem}[77] Let $x,y$ be sets. Then $y^x$ is a set.
\end{theorem}
\begin{proof} $y^x \subset 2^{x \times y}$.
\end{proof}

\begin{definition}[78] $f$ is \emph{on} $x$ iff $x = {\rm domain} f$.
\end{definition}

\begin{definition}[79] $f$ is \emph{to} $y$ iff ${\rm range} f \subset y$.
\end{definition}

\begin{definition}[80] $f$ is \emph{onto} $y$ iff ${\rm range} f = y$.
\end{definition}

\end{forthel}






\end{document}